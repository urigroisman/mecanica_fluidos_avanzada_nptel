
\documentclass{article}
\usepackage{amsmath, amssymb}
\usepackage{physics}
\usepackage{geometry}
\geometry{margin=1in}

\begin{document}

\section*{Vector Calculus and Differential Equations Problems}

\subsection*{Problem 1: Coplanarity of Three Vectors}

The three vectors \( \vec{a}, \vec{b}, \vec{c} \) are coplanar if and only if their scalar triple product is zero:
\[
\vec{a} \cdot (\vec{b} \times \vec{c}) = 0
\]

This condition ensures the volume of the parallelepiped formed by the vectors is zero, i.e., they lie in the same plane.

\noindent\textbf{Final Answer: } \fbox{$\vec{a} \cdot (\vec{b} \times \vec{c}) = 0$}

\bigskip

\subsection*{Problem 2: Tangent Plane to a Surface}

We are given the surface:
\[
z = 2e^{-x} \sin(2y)
\]
and the point \( \left(0, \frac{\pi}{12}, 1\right) \). To find the tangent plane, we compute:

\[
f_x = -2e^{-x} \sin(2y), \quad f_y = 4e^{-x} \cos(2y)
\]

At \( (0, \frac{\pi}{12}) \), we evaluate:
\[
f_x = -2 \cdot \sin\left(\frac{\pi}{6}\right) = -1, \quad f_y = 4 \cdot \cos\left(\frac{\pi}{6}\right) = 2\sqrt{3}
\]

Using the tangent plane formula:
\[
z - 1 = -x + 2\sqrt{3} \left(y - \frac{\pi}{12} \right)
\Rightarrow z = -x + 2\sqrt{3}y + \left(1 - \frac{\sqrt{3} \pi}{6} \right)
\]

Rewriting:
\[
x - 2\sqrt{3}y + z = \frac{6 - \sqrt{3}\pi}{6}
\]

\noindent\textbf{Final Answer: } \fbox{$x - 2\sqrt{3}y + z = \frac{-\sqrt{3} \pi + 6}{6}$}

\end{document}
