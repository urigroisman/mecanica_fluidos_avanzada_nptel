
\documentclass[12pt]{article}
\usepackage{amsmath}
\usepackage{geometry}
\geometry{margin=2.5cm}
\title{Week 02 -- Problem 9: No-Slip Condition and Euler's Equations}
\date{}
\begin{document}
\maketitle

\section*{Problem}

\textit{Why cannot the no-slip boundary condition be used when solving Euler’s equations for fluid flow?}

\begin{itemize}
\item[(a)] Because the flow is compressible
\item[(b)] Because Euler’s equations do not account for body forces
\item[(c)] Because Euler’s equations neglect viscosity
\item[(d)] Because Euler’s equations assume irrotational flow
\end{itemize}

\section*{Analysis}

Euler’s equations describe inviscid fluid flow:

\[
\rho \left( \frac{\partial \vec{v}}{\partial t} + (\vec{v} \cdot \nabla)\vec{v} \right) = -\nabla p + \rho \vec{g}
\]

These equations **neglect viscosity**.

The no-slip boundary condition arises from viscous effects. It enforces:

\[
\vec{v}_{\text{fluid}} = \vec{v}_{\text{wall}}
\]

Such a condition cannot be physically justified when viscosity is zero. Therefore, it is incompatible with the Euler formulation.

\section*{Evaluation of Options}

\begin{itemize}
\item[(a)] Irrelevant to no-slip — compressibility is unrelated.
\item[(b)] Incorrect — Euler equations \textit{do} account for body forces.
\item[(c)] \textbf{Correct} — no-slip requires viscosity.
\item[(d)] Incorrect — Euler equations do not inherently assume irrotationality.
\end{itemize}

\section*{Final Answer}

\[
\boxed{\text{(c) Because Euler’s equations neglect viscosity}}
\]

\end{document}
