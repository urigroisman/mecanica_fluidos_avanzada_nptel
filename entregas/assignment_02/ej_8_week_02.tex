
\documentclass[12pt]{article}
\usepackage{amsmath}
\usepackage{geometry}
\geometry{margin=2.5cm}
\title{Week 02 -- Problem 8: Bernoulli Equation in Entrance Flow Region}
\date{}
\begin{document}
\maketitle

\section*{Problem}

For a steady, incompressible flow between two infinite parallel plates, points \( p \) and \( q \) lie close to one another near the centerline in the entrance (core) region.

Which of the following statements about Bernoulli’s equation is true?

\section*{Option Analysis}

\subsection*{(a) Bernoulli can be applied because viscous effects are negligible}

\textbf{True.} Near the centerline in the entrance region, viscous diffusion has not reached, so the flow behaves inviscidly. Bernoulli’s equation is valid in such regions.

\subsection*{(b) Bernoulli is not applicable between \( p \) and \( q \)}

\textbf{False.} Viscous effects are negligible in the core, hence Bernoulli \emph{is} applicable.

\subsection*{(c) Bernoulli is applicable since \( p \) and \( q \) lie along a streamline}

\textbf{True.} If \( p \) and \( q \) are close and lie along a streamline in a steady, incompressible, inviscid region, then Bernoulli can be applied along the streamline.

\subsection*{(d) Both (a) and (c)}

\textbf{Correct.} Both reasoning in (a) and (c) are valid and justify the use of Bernoulli’s equation.

\section*{Final Answer}

\[
\boxed{\text{(d) Both (a) and (c)}}
\]

\end{document}
