
\documentclass[12pt]{article}
\usepackage{amsmath}
\usepackage{geometry}
\geometry{margin=2.5cm}
\title{Week 02 -- Problem 7: Reynolds Transport Theorem (RTT)}
\date{}
\begin{document}
\maketitle

\section*{Problem Statement}

Consider the following statements:

\begin{enumerate}
    \item RTT is valid for both solids and fluids.
    \item RTT is valid only for fluids.
    \item RTT for conservation of linear momentum makes use of Newton’s second law.
    \item The given form of RTT is valid for a non-deformable control volume.
\end{enumerate}

We are given:

\[
\frac{dm}{dt}\Big|_{\text{sys}} = \int_{\text{CV}} \frac{\partial \rho}{\partial t} \, dV + \int_{\text{CS}} \rho \vec{V} \cdot \vec{n} \, dA
\]

\section*{Analysis}

\subsection*{Statement 1: True}

RTT is derived from general mathematical principles and is applicable to both solids and fluids.

\subsection*{Statement 2: False}

RTT is not limited to fluids. It applies to any continuum.

\subsection*{Statement 3: False (Ambiguous)}

RTT applied to linear momentum gives:

\[
\frac{d}{dt} \int_{\text{sys}} \rho \vec{V} \, dV = \frac{d}{dt} \int_{\text{CV}} \rho \vec{V} \, dV + \int_{\text{CS}} \rho \vec{V} (\vec{V} \cdot \vec{n}) \, dA
\]

This is purely a transformation. Newton's second law is a **separate physical postulate** used **after** RTT, when equating momentum change to force. Hence, this statement is misleading.

\subsection*{Statement 4: True}

The integral form shown assumes a fixed, non-deforming control volume. Therefore, it is valid only under those conditions.

\section*{Final Answer}

\[
\boxed{\text{(a) 1 and 4}}
\]

\end{document}
